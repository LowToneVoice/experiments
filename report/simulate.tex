\documentclass[dvipdfmx]{jsarticle}
\usepackage{p2report}

\begin{document}

\section{数値計算による実験方法の検討}

実験に先立って
\begin{itemize}
    \item 波動関数の直接計算による確率分布の計算
    \item 確率分布に基づくモンテカルロシミュレーション
\end{itemize}
を行った。

% \begin{figure}
%     \centering
%     \begin{tikzpicture}
%         \tikzset{Terminal/.style={rounded rectangle,  draw,  text centered, text width=3cm, minimum height=1.5cm}};
%         \tikzset{Process/.style={rectangle,  draw,  text centered, text width=3cm, minimum height=1.5cm}};
%         \tikzset{Decision/.style={diamond,  draw,  text centered, aspect=3,text width=5cm, minimum height=1.5cm}};
%         \node[Terminal](a)at (0,0){Start};
%         \node[Process, below=1.5 of a.center](b){Process};
%         \draw[->, thick]  (a) --(b);
%         \node[Process, below=2 of b.center](c){Process};
%         \draw[->, thick]  (b) --(c);
%         \node[Decision](d)at (0,-7.5){Judge};
%         \draw[->, thick]  (c) --(d);
%         \node[Process, above left=of d ](da){Process};
%         \draw[->,thick]  (d)node[below, xshift=-100]{No} -| (da);
%         \draw[->,thick]  (da) |- (0,-3.5);
%         \node[Process, below=1.75 of d.center](e){1st gate};
%         \draw[->, thick]  (d) --(e) node[right,yshift=30]{Yes};
%         \node[Decision, below=1.35 of e.center](f){Judge 2};
%         \draw[->, thick]  (e) --(f);
%         \node[Process, above right=of f](fa){Back};
%         \draw[->, thick]  (f)node[below, xshift=100]{No}  -| (fa);
%         \draw[->,thick]  (fa) |-(0,-3.5);
%         \node[Process, below=1.75 of f.center](g){2nd gate};
%         \draw[->, thick]  (f) --(g)node[right,yshift=30]{Yes};
%         \node[Terminal, below=1.75 of g.center](h){End};
%         \draw[->, thick]  (g) --(h);
%     \end{tikzpicture}
% \end{figure}


\subsection{波動関数の直接計算}

\eqref{eq: theoretical: R and T}を用いて図\ref{fig: theoretical: alpha and paths}点EにおけるHビーム、Oビームの波動関数を計算、強度を求めると図のようになった。


\subsection{モンテカルロシミュレーション}

波動関数に従ってモンテカルロシミュレーションを行い、

\end{document}
