\documentclass[dvipdfmx]{jsarticle}
\usepackage{p2report}

\begin{document}

\section{数値計算による実験方法の検討}

実験に先立ってモンテカルロシミュレーションを行った。

\begin{figure}
    \centering
    \begin{tikzpicture}
        \tikzset{Terminal/.style={rounded rectangle,  draw,  text centered, text width=3cm, minimum height=1.5cm}};
        \tikzset{Process/.style={rectangle,  draw,  text centered, text width=3cm, minimum height=1.5cm}};
        \tikzset{Decision/.style={diamond,  draw,  text centered, aspect=3,text width=5cm, minimum height=1.5cm}};
        \node[Terminal](a)at (0,0){Start};
        \node[Process, below=1.5 of a.center](b){ぷろせす};
        \draw[->, thick]  (a) --(b);
        \node[Process, below=2 of b.center](c){ぷろせす};
        \draw[->, thick]  (b) --(c);
        \node[Decision](d)at (0,-7.5){はんだんするぞ!};
        \draw[->, thick]  (c) --(d);
        \node[Process, above left=of d ](da){なんらかのぷろせす};
        \draw[->,thick]  (d)node[below, xshift=-100]{No} -| (da);
        \draw[->,thick]  (da) |- (0,-3.5);
        \node[Process, below=1.75 of d.center](e){第一関門クリア!};
        \draw[->, thick]  (d) --(e) node[right,yshift=30]{Yes};
        \node[Decision, below=1.35 of e.center](f){はんだんその2};
        \draw[->, thick]  (e) --(f);
        \node[Process, above right=of f](fa){もどる};
        \draw[->, thick]  (f)node[below, xshift=100]{No}  -| (fa);
        \draw[->,thick]  (fa) |-(0,-3.5);
        \node[Process, below=1.75 of f.center](g){第二関門突破!};
        \draw[->, thick]  (f) --(g)node[right,yshift=30]{Yes};
        \node[Terminal, below=1.75 of g.center](h){End};
        \draw[->, thick]  (g) --(h);
    \end{tikzpicture}
\end{figure}

\end{document}
